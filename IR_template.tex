% !TEX program = xelatex -> bibtex -> xelatex*2
\documentclass[twocolumn]{article}
    \usepackage{cite}
    \usepackage{flushend,cuted}
    \usepackage{geometry}
    \usepackage{fontspec}
    \usepackage{titlesec}

% 调整页边距
\geometry{a4paper, left=2cm,right=2cm,top=1cm,bottom=2cm}

% 定义要使用的字体,FE为论文题目所需字体
\newcommand{\FE}{\fontspec{FagoNoTf ExtraBold}}
\newcommand{\TF}{\fontspec{TransitFront}}

% 设置正文使用的字体
\setmainfont{TerminusSSK}

% 设置各级标题用到的字体
\titleformat*{\section}{\large\FE}
\titleformat*{\subsection}{\normalsize\bfseries\TF}

% 设置上下文间距
\titlespacing*{\chapter} {0pt}{50pt}{40pt}
\titlespacing*{\section} {0pt}{3.5ex plus 1ex minus .2ex}{2.3ex plus .2ex}
\titlespacing*{\subsection} {0pt}{3.25ex plus 1ex minus .2ex}{1.5ex plus .2ex}
\titlespacing*{\subsubsection}{0pt}{3.25ex plus 1ex minus .2ex}{1.5ex plus .2ex}
\titlespacing*{\paragraph} {0pt}{3.25ex plus 1ex minus .2ex}{1em}
\titlespacing*{\subparagraph} {\parindent}{3.25ex plus 1ex minus .2ex}{1em}

% 正文部分
%%%%%%%%%%%%%%%%%%%%%%%%%%% 正文将包含六个部分 %%%%%%%%%%%%%%%%%%%%%%%%%%%%
% abstract                                                              %
% 1. Introduction                                                       %
% 2. Design Goals/Related work                                          %
% 3. Robot Design Specifications                                        %
% 4. Planning and navigation                                            %
% 5. Image Processing                                                   %
% 6. Testing and Experimental Results                                   %
%%%%%%%%%%%%%%%%%%%%%%%%%%% 正文将包含六个部分 %%%%%%%%%%%%%%%%%%%%%%%%%%%%

\begin{document}
% 修改论文题目
% Industrial Robot期刊要求文件不能有作者信息,所以这里只需要准备论文题目即可
% 题目使用FagoNoTf ExtraBold字体,加粗,huge,与摘要的距离为-1em
\title{\huge{\FE Development and Implementation of a Mobile Robot for Road Crack Detection
\vspace{-1em}}}
% 日期为空,则不显示日期信息
\author{}
\date{}
\maketitle
%%%%%%%%%%%%%%%%%%%%%%%%%%%%%%%%% 摘要 %%%%%%%%%%%%%%%%%%%%%%%%%%%%%%%%%%%
%%% 摘要:(最多250字)                                                   %
%       1. Purpose(目的)------------------------------------- 必需     %
%       2. Design/methodology/approach(方法)----------------- 必需     %
%       3. Findings(发现)------------------------------------ 必需     %
%       4. Research limitations/implications(限制/启示)------ 非必需    %
%       5. Practical implications(现实意义)------------------ 非必需    %
%       6. Social implications(社会影响)--------------------- 非必需    %
%       7. Originality/value(创意/价值)---------------------- 必需      %
%%%%%%%%%%%%%%%%%%%%%%%%%%%%%%%%% 摘要 %%%%%%%%%%%%%%%%%%%%%%%%%%%%%%%%%%%
    \begin{strip}
        {\FE\normalsize{Abstract}}\\
        {\TF{Prupose}} - Please type your text here.\\
        {\TF{Design/methodology/approach}} - Please type your text here.\\
        {\TF{Findings}} - Please type your text here.\\
        {\TF{Research limitations/implications}} - Please type your text here.\\
        {\TF{Practical implications}} - Please type your text here.\\
        {\TF{Social implications}} - Please type your text here.\\
        {\TF{Originality/value}} - Please type your text here.\\
    \end{strip}
    
    \section{Introduction}
        Lorem ipsum dolor sit amet, consectetur adipisicing elit, sed do eiusmod
        tempor incididunt ut labore et dolore magna aliqua. Ut enim ad minim veniam,
        quis nostrud exercitation ullamco laboris nisi ut aliquip ex ea commodo
        consequat.Duis aute irure dolor in reprehenderit in voluptate velit esse
        cillum dolore eu fugiat nulla pariatur. Excepteur sint occaecat cupidatat non
        proident, \cite{GraffZivin2018} sunt in culpa qui officia deserunt mollit anim id est laborum.

    \section{Design Goals}
    I believe Kendall wanted the American Idol audition so much that she willed herself to move again. One of her friends brought a microphone to the hospital and put it on her bed. Every day, Kendall tried hard to pick it up with her right hand. It was more important for her to pick up that mic than a spoon or fork.
    
    \section{Robot Design Specifications}
        Lorem ipsum dolor sit amet, consectetur adipisicing elit, sed do eiusmod
        tempor incididunt ut labore et dolore magna aliqua. Ut enim ad minim veniam,
        quis nostrud exercitation ullamco laboris nisi ut aliquip ex ea commodo
        consequat.Duis aute irure dolor in reprehenderit in voluptate velit esse
        cillum dolore eu fugiat nulla pariatur. Excepteur sint occaecat cupidatat non
        proident, sunt in culpa qui officia deserunt mollit anim id est laborum.
    
    \section{The robot motion scheme}
    \subsection{Manual remote control}
    The robot can be remote control by joystick or the graphic user interface(GUI) on the touch screen. 
    (Figure 1). For the joystick control, instruction data is transmitted to the robot’s onboard chip via 
    2.4Ghz RF. There are two rockers on the joystick, which correspond to the plane linear velocity of 
    the robot and the angular velocity of the wheel. With the cooperation of the two rockers, the robot 
    can make complex movements in any directions on the plane. This way has good performance in offline 
    environment because it’s simple and stable nature. When the network is allowed, users can control the 
    robot through the GUI on the touch screen, as shown in (Figure 2). There are ten motion modes for 
    users to choose, the button on the bottom is used to adjust the plane linear velocity and rotational 
    angular velocity of the robot. One advantage of using the screen is that the user can observe the 
    current speed feedback of the robot and the steering angle of each wheel in real time. The GUI scheme 
    is suitable for use in situations where simple control is required but feedback is necessary.
    \subsection{Autonomous navigation}
    In certain circumstances, robot needs to act autonomously or inspect uninterruptedly, manual remote 
    control is not the best option for these kinds of tasks. So, Jupiter is also designed to has the 
    ability to navigate automatically. The framework of the navigation system is shown in (Figure 3), 
    with the lasers mounted on the opposite corner of the robot, Jupiter can receive environmental 
    information which can be used to locate and map. Considering that the laser used is high in frequency 
    and there are no other sensors such as cameras and odometers installed, the SLAM algorithm used here 
    is Hector SLAM. Laser data is utilized to update the current pose of the robot and the representation 
    of the map, when the start and target points are determined, the Dijkstra algorithm will be used to 
    generate a feasible path connecting two points. In the process of autonomous navigation, the dynamic 
    obstacles will appear on the map in real time in form of occupied grids. Jupiter will use the Dynamic 
    Window Approach to re-plan the feasible path to the target point based on the location and the size 
    of the obstacles. The robot will stand still when it reaches the target point until the next target 
    order is release.
    
    \section{Image Processing}
        Lorem ipsum dolor sit amet, consectetur adipisicing elit, sed do eiusmod
        tempor incididunt ut labore et dolore magna aliqua. Ut enim ad minim veniam,
        quis nostrud exercitation ullamco laboris nisi ut aliquip ex ea commodo
        consequat.Duis aute irure dolor in reprehenderit in voluptate velit esse
        cillum dolore eu fugiat nulla pariatur. Excepteur sint occaecat cupidatat non
        proident, sunt in culpa qui officia deserunt mollit anim id est laborum.
     
     \section{Testing and Experimental Results}
        Lorem ipsum dolor sit amet, consectetur adipisicing elit, sed do eiusmod
        tempor incididunt ut labore et dolore magna aliqua. Ut enim ad minim veniam,
        quis nostrud exercitation ullamco laboris nisi ut aliquip ex ea commodo
        consequat.Duis aute irure dolor in reprehenderit in voluptate velit esse
        cillum dolore eu fugiat nulla pariatur. Excepteur sint occaecat cupidatat non
        proident, sunt in culpa qui officia deserunt mollit anim id est laborum.
    
    \bibliography{info}
    \bibliographystyle{plainnat}
\end{document}
